\chapter{Conclusion}
In this thesis a classification pipeline has been developed to detect 14 different finger gestures using solely EMG signals. This chapter provides a summary of this thesis and explains the reasons behind the choices that were made. Future suggestions to improve the existed EMG method are also included in the section 5.1.\\
To build an ideal EMG based gesture recognition system should consist of an EMG signal analysis system which is precise, quick and credible. However, the purpose of this thesis is not to develop a complete functional system but rather to explore the performance of a feed forward neural network when its input is consisting of different extracted features from the EMG signals. \\ 
In this work, we proposed an EMG method for the classification of finger gestures from intact subjects. For the classification we decided to experiment with a feedforward neural network. In order to adjust correctly the parameters for our neural network we used as initial input EMG signals from my forearm for the flexion and the extension of the thumb with the palm facing upwards.\\
The motive of this work is to examine a majority of possibilities of using machine learning to classify EMG data. This work only supports the simplest neural network architecture, the fully connected feed forward neural networks. In this work we experiment with one kind of network architecture the Multi-layer Perceptron with the focus to determine which parameters are of great significance to the accuracy and possibly performance. \\
We decided to follow a hybrid approach of the forward selection, backward elimination. We started training a neural network of one hidden layer and 5 hidden neurons with all the possible duplets of the existing features. After the training, we tested also the trained neural networks and we got the detection accuracies. From the detection accuracies we got the 3 best performances. We combined each of the best three duplets with the rest features and trained another neural network with one hidden layer and five hidden neurons. The duplet with the best detection performance was chosen and we used that duplet to combine with the rest features forming triplets. With each triplet we trained feed-forward neural networks with one hidden layer and five hidden neurons but this time incrementing the number of hidden layers by one. \\
The limitation of time did not allow us to experiment with other parameters than the number of hidden layers. The maximum number of hidden layers we reached was 13. For decided for the evaluation of the data we would a number of hidden layers which would not be close to the maximum number of layers in order to avoid possible overfitting of the evaluation data. We decided to choose  8 hidden layers for the final neural network. For that decision the Fig. \ref{fig:sym4sym8} helped us to come to the conclusion that the beginning of the plateau of the accuracies started to form around for 8 hidden layers. For that specific number of hidden layers we noted which triplets gave the best detection accuracies. \\
For the evaluation stage we created a database consisting solely of EMG signals coming from 14 finger gestures of 21 intact subjects. Each finger gesture is consisting of one flexion and one extension. As aforementioned,  the final neural network had 8 hidden layers with 5 hidden neurons each. Furthermore, the final training did not happen with all the possible triplets but only for the best performed during the process of the parameter adjustment.  For each finger gesture a classifier was built in order to distinguish between the extension and the flexion and the final corresponding detection accuracy was noted. \\
Based on the results we received, we believe it is difficult to distinguish delicate finger gestures only with EMG data and that classifiers would perform poorly if we are to compare them with other vision techniques or wearable gloves. Although, the importance of the results they make us believe that EMG data could be used as an addition to systems that also depend on other types of data. One of the most significant characteristics of the Myo armband is that its EMG sensors are portable. Regarding the fact that the basis for a successful gesture recognition system are portability and convenience we believe that a hybrid approach using both visual and EMG data could be robust, compact and considerably less cumbersome than wearable gloves.
\section{Future Work}
In this section we will provide potential improvements to the current pipeline.
\subsection{Size of data}
Creating a huge dataset to include all the relevant variations helps us to resolve to a great extent the problem of considerable changes in EMG signals from person to person, as we already mentioned. This will also improve the robustness of the EMG signals regardless the potential changes of the Myo's position on the subject's arm. 
\subsection{Amount of features and reduction of the feature vector}
For this work, we used only 11 features to extract useful information. We recommend for the further exploration of the EMG signals the use of more features on them that performed well on other works similar to ours but did not apply them on this work. For example, for the further adjustment of the preprocessing phase we recommend:
\begin{itemize}
\item to experiment with the window size for the windowing part.
\item test more discrete wavelets and different decompositiong levels.
\item further experiment with the continuous wavelet transform as an alternative to DWT.
\end{itemize}
What's more, the size of the final feature vector which we used was big enough and this leads to a long run time. For the future exploration of the existing finger gesture database we recommend the application of a method which has as purpose the reduction of the dimension of the final feature vector such as the Principal Component Analysis (PCA) or Linear Discriminant  Analysis (LDA).
\subsection{Network Architectures}
This work is based on a simple feed forward network architecture. We encourage the further experimentation with other kinds of network architectures. In case we keep the time information along with the amplitude of the EMG signal we can use the Recurrent Neural Network architecture (RNN) which has the ability to remember and make use of the sequential information. This feature can be useful for analyzing a sequence of gestures.\\
Another deep learning architecture which is recently widely used achieving a considerable performance is the Convolutional Neural Network (CNN) architecture \cite{schmidhuber_deep_2015, he_deep_2016}. Even though CNN architecture is used in image analysis, it has also its applications on time-series data, such as speech recognition \cite{qian_very_2016} and gesture recognition \cite{atzori_deep_2016} . 