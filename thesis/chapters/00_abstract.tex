\cleardoublepage
\vspace {2cm}
\begin{center}
\paragraph{Abstract}
\hrulefill
\end{center}
Wearable devices have been evolving over the last years. The integration of sensors on them encourages scientists to use wearable devices more frequently as they are cheaper and the purchasing accessibility is easier than other specialized devices. In this work, we will use a gesture control armband called Myo. Myo is equipped with a gyroscope, magnetometer, accelerometer and a set of electromyographic (EMG) sensors. Along with the Myo we will use a cell phone which receives the signals from the Myo. The focus point of this master thesis is to research different features based only on EMG signals coming from finger movements. EMG is a signal measurement of the electrical activity of the muscles and is one of the most common sources of information used to study muscle function and neurological disorders. One of the main reasons behind that is that we want to correctly detect finger movements using only the EMG signals that are associated only with arm muscles responsible for motoring the fingers. Also, the existing literature is limited on finger gesture recognition using non-intruding device such as Myo \cite{myo}. The goal of this master thesis is to apply the earned knowledge to Myo controlled prosthetic arms worn by amputees that have lost some or all of their fingers or patients that have deficiency of moving their fingers due to a stroke, tendinitis, or peripheral neuropathy. The fact that each finger has a different degree of freedom (DoF) gives us the chance to detect various finger gestures. The most trite ones which we will examine are the extension and the flexion of a finger or a group of fingers. In particular, the thumb can be flexed and extended (2 DoF). The rest of the fingers can be flexed and extended in one direction (1 DoF). The need of a solid classification between these two different kind of movements is imperative. What’s more, there can be more combinations such as holding three fingers (thumb excluded) as a cluster and moving forward backward the remaining free finger. The increasing degree of complexity of the different finger gestures plays a significant role in order to have a more complete detection cover over the whole spectrum of finger gestures. After the collection of the recordings, we will introduce a preprocessing stage which will help us in extracting features from the raw data. In this work, we will use a range of methods such as root mean square (RMS), moving average (smooth), Short-Time Fourier Transform (STFT) and a variety of discrete wavelet transforms. The collection of all feature extraction vectors will then be passed into classifiers and classification errors from each feature will be compared with each other in order to determine the best performance. In the implemented framework we will examine the approach of an \ac{ANN} classifier.
